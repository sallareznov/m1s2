\section*{Introduction}
Ce TP est une implémentation en JAVA d'une passerelle REST communiquant avec un serveur FTP.

\section*{Listing des dossiers et fichiers du projet}
\begin{description}
	\item[doc/ :] javadoc du projet
	\item[lib/ :] contient les librairies (\verb+.jar+) dont le projet est dépendant
	\item[src/ :] contient les fichiers sources (\verb+.java+) du projet
	\item[test/ :] contient les fichiers de tests
	\item[uml/ :] contient les diagrammes de classes du projet
	\item[build.xml :] fichier de gestion de projet Ant
	\item[client.jar :] archive exécutable du client
	\item[java.policy :]
	\item[rmi.eps :] schéma général de rmi
	\item[runclient.sh :] script de lancement d'un client
	\item[runserver.sh :] script de lancement d'un serveur
	\item[server.jar :] archive exécutable du serveur
	\item[sonar-project.properties :] fichier .properties pour le contrôle qualité Sonar
\end{description}

\section*{Utilisation}
	Pour lancer l'exécutable, deux solutions sont possibles :\\
		\underline{Via Ant}
			\begin{itemize}
				\item \verb+ant compile+
				\item \verb+ant runserver+ puis \verb+java -jar restBridge.jar+ 
			\end{itemize}
		\underline{}

\section*{Architecture}
\begin{tabbing}
	\hspace{1cm}\=\hspace{1cm}\=\kill
	\textit{- package rest}\\
		\>\textit{- package config}\\
			\>\>\verb+AppConfig+ : classe contenant la configuration de l'application\\
		\>\textit{- package logger}\\
			\>\>\verb+FTPLoggerFactory+ : factory de loggers\\
			\>\>\verb+FTPLoggerSimpleFormatter+ : formatter qui nous abstient du sucre syntaxique des \\\>\>loggers habituels \\
		\>\textit{- package model}\\
			\>\>\verb+FileServlet+ : servlet de fichiers (typiquement pour les icônes)\\
			\>\>\verb+FTPClientFactory+ : factory chargée de créer des instances de FTPClient\\
			\>\>\verb+FTPRestServiceConfiguration+ : configuration du service\\
			\>\>\verb+ItemBuilder+ : builder d'items d'un répertoire donné (en HTML)\\
		\>\textit{- package rs}\\
			\>\>\verb+FTPRestService+ : service REST-FTP\\
			\>\>\verb+JaxRsApiApplication+\\
		\>\textit{- package services}\\
			\>\>\verb+FTPRestService+ : service REST\\	
	\>\verb+Starter+ : classe principale
\end{tabbing}
